% Options for packages loaded elsewhere
\PassOptionsToPackage{unicode}{hyperref}
\PassOptionsToPackage{hyphens}{url}
%
\documentclass[
]{article}
\usepackage{lmodern}
\usepackage{amssymb,amsmath}
\usepackage{ifxetex,ifluatex}
\ifnum 0\ifxetex 1\fi\ifluatex 1\fi=0 % if pdftex
  \usepackage[T1]{fontenc}
  \usepackage[utf8]{inputenc}
  \usepackage{textcomp} % provide euro and other symbols
\else % if luatex or xetex
  \usepackage{unicode-math}
  \defaultfontfeatures{Scale=MatchLowercase}
  \defaultfontfeatures[\rmfamily]{Ligatures=TeX,Scale=1}
\fi
% Use upquote if available, for straight quotes in verbatim environments
\IfFileExists{upquote.sty}{\usepackage{upquote}}{}
\IfFileExists{microtype.sty}{% use microtype if available
  \usepackage[]{microtype}
  \UseMicrotypeSet[protrusion]{basicmath} % disable protrusion for tt fonts
}{}
\makeatletter
\@ifundefined{KOMAClassName}{% if non-KOMA class
  \IfFileExists{parskip.sty}{%
    \usepackage{parskip}
  }{% else
    \setlength{\parindent}{0pt}
    \setlength{\parskip}{6pt plus 2pt minus 1pt}}
}{% if KOMA class
  \KOMAoptions{parskip=half}}
\makeatother
\usepackage{xcolor}
\IfFileExists{xurl.sty}{\usepackage{xurl}}{} % add URL line breaks if available
\IfFileExists{bookmark.sty}{\usepackage{bookmark}}{\usepackage{hyperref}}
\hypersetup{
  pdftitle={Relatório técnico de dados do PURECO},
  pdfauthor={Elizabeth Borgognoni Souto},
  hidelinks,
  pdfcreator={LaTeX via pandoc}}
\urlstyle{same} % disable monospaced font for URLs
\usepackage[margin=1in]{geometry}
\usepackage{graphicx,grffile}
\makeatletter
\def\maxwidth{\ifdim\Gin@nat@width>\linewidth\linewidth\else\Gin@nat@width\fi}
\def\maxheight{\ifdim\Gin@nat@height>\textheight\textheight\else\Gin@nat@height\fi}
\makeatother
% Scale images if necessary, so that they will not overflow the page
% margins by default, and it is still possible to overwrite the defaults
% using explicit options in \includegraphics[width, height, ...]{}
\setkeys{Gin}{width=\maxwidth,height=\maxheight,keepaspectratio}
% Set default figure placement to htbp
\makeatletter
\def\fps@figure{htbp}
\makeatother
\setlength{\emergencystretch}{3em} % prevent overfull lines
\providecommand{\tightlist}{%
  \setlength{\itemsep}{0pt}\setlength{\parskip}{0pt}}
\setcounter{secnumdepth}{-\maxdimen} % remove section numbering

\title{Relatório técnico de dados do PURECO}
\author{Elizabeth Borgognoni Souto}
\date{10/12/2020}

\begin{document}
\maketitle

\hypertarget{trabalhando-com-os-dados-descriuxe7uxe3o-dos-dados-comentuxe1rios-sobre-formas-de-padronizauxe7uxe3o-e-sobre-as-planilhas-coletadas-pelo-aplicativo.}{%
\subsubsection{Trabalhando com os dados: Descrição dos dados,
comentários sobre formas de padronização e sobre as planilhas coletadas
pelo
aplicativo.}\label{trabalhando-com-os-dados-descriuxe7uxe3o-dos-dados-comentuxe1rios-sobre-formas-de-padronizauxe7uxe3o-e-sobre-as-planilhas-coletadas-pelo-aplicativo.}}

Explicação das variavéis da planilha de faxinas:

\begin{itemize}
\tightlist
\item
  Tabela \textbf{Faxinas}: período de 2018 a 2020.

  \begin{itemize}
  \item
    \textbf{Data}: Data marcada da faxina.
  \item
    \textbf{Mulher}: Nome da mulher que realizou a faxina.
  \item
    \textbf{Valor}: Valor cobrado da faxina em reais.
  \item
    \textbf{Cliente}: Nome do cliente que pediu a faxina.
  \item
    \textbf{Endereço}: Endereço do cliente que pediu a faxina.
  \item
    \textbf{Ocorreu?}: Representa se a faxina marcada na data ocorreu.
  \item
    \textbf{Feedback Colhido?}: Representa se o feedback do cliente foi
    anotado.
  \item
    \textbf{Onde foi Colhido?}: Local em que o feedback foi anotado.
  \item
    \textbf{Feedback Cliente}: Nota que o cliente deu para a faxina.
  \item
    \textbf{Nota feedback Mulher}: Nota que a mulher deu para o seu
    cliente.
  \item
    \textbf{Nota feedback Cliente}: Nota que o cliente deu para a
    faxina.
  \item
    \textbf{Feedback Cliente}: Comentários dos clientes sobre a faxina.
  \item
    \textbf{Comentários}: Comentários gerais sobre os serviços
    realizados.
  \item
    \textbf{Remarcou} : Representa se a faxina foi remarcada.
  \item
    \textbf{Tipo}: Representa se o cliente já havia pedido antes uma
    faxina ou se um pedido de faxina novo.
  \end{itemize}
\item
  Tabela \textbf{Disponibilidade}: período de 2018 a 2020.

  \begin{itemize}
  \item
    \textbf{Mulher} : Nome da mulher que tem disponibilidade para
    realizar a faxina.
  \item
    \textbf{Disponibilidade} : Disponibilidade do número de faxinas que
    pode realizar por dia.
  \item
    \textbf{Data} : Data disponível para a faxina.
  \end{itemize}
\end{itemize}

Mudanças na etapa de limpeza dos dados:

\begin{itemize}
\item
  Usou-se o comando join para juntar as duas planilhas do período de
  2018 a 2019 e de 2019 a 2020 em apenas uma.
\item
  Números da coluna \textbf{Valor}: removeu-se os \textbf{R\$} da frente
  dos números.
\item
  Nome dos endereços: Modificou-se para quando for uma Rua começar com
  apenas \emph{Rua} e quando for Avenida começar com \emph{Avenida}.
\item
  Valores como \textbf{2,5 \textasciitilde{} 3,0} foram truncados para o
  máximo \textbf{3,0}.
\item
  Nome das colunas \textbf{Feedback Cliente} e \textbf{Feedback Mulher}
  da planilha mais antiga modificados para \textbf{Nota feedback
  Cliente} e \textbf{Nota feedback Mulher}.
\item
  Removeu-se todas as colunas vazias ou que continham números sem
  sentido.
\item
  Os valores: 0 ou 1, true ou false, que se transformaram em Sim ou Não.
\end{itemize}

Ideias de análises e gráficos:

\begin{itemize}
\item
  Quantidade de faxinas por Dia da semana - OK
\item
  Quantidade de faxinas por Tipo: o gráfico mostra se há mais clientes
  novos requisitando faxinas ou se a maioria ainda são os clientes mais
  antigos. - OK
\item
  Média de valor da faxina por mulher.
\item
  Tabelas - sumários estatísticos.
\item
  Mapas, acrescentar latitude e longitude de acordo com os endereços
  (\emph{ver pacote rselenion})
\item
  Porcentagem de faxinas que não ocorreram, que foram remarcadas ou não.
\item
  Gráfico do valor ganho com as faxinas por mês para cada faxineira.
\item
  Análise dos comentários positivos. (\emph{ver se e possível})
\end{itemize}

\begin{verbatim}
## # A tibble: 10 x 11
##    Data  Mulher Valor Cliente Endereço `Ocorreu?` `Feedback Colhi~
##    <chr> <chr>  <dbl> <chr>   <chr>    <chr>      <chr>           
##  1 03/0~ Lourd~    80 Eloise~ Prof Du~ Sim        Não             
##  2 12/0~ Zilza    130 Prof S~ <NA>     Sim        Sim             
##  3 22/0~ Lourd~   170 Derick  <NA>     Sim        Sim             
##  4 08/0~ Zilza    150 Laís Z~ <NA>     Sim        Sim             
##  5 12/0~ Lourd~    80 <NA>    <NA>     Sim        Sim             
##  6 16/0~ Zilza    130 <NA>    <NA>     Sim        Sim             
##  7 17/0~ Zilza    130 <NA>    <NA>     Sim        Sim             
##  8 18/0~ Lourd~    80 <NA>    <NA>     Sim        Sim             
##  9 18/0~ Lourd~    80 <NA>    <NA>     Sim        Sim             
## 10 23/0~ Zilza    150 <NA>    <NA>     Sim        Sim             
## # ... with 4 more variables: `Onde foi colhido?` <chr>, `Feedback
## #   cliente` <dbl>, `Feedback mulher` <dbl>, Comentários <chr>
\end{verbatim}

\begin{verbatim}
## # A tibble: 10 x 15
##    Data  `Ocorreu?` Remarcou Mulher Valor Cliente Tipo  Endereço Região
##    <chr> <chr>      <chr>    <chr>  <dbl> <chr>   <chr> <chr>    <chr> 
##  1 02/0~ Não        Não      Zilza    170 Wesley~ Já h~ Rua  Dr~ Barão~
##  2 02/0~ Sim        Sim      Lourd~   150 Thiago~ Já h~ Rua  Ro~ Barão~
##  3 02/0~ Sim        Não      Vilan~   100 Eloise~ Já h~ Prof Du~ Barão~
##  4 03/0~ Sim        Não      Vilan~   130 Karim ~ Já h~ Arthur ~ Barão~
##  5 04/0~ Sim        Não      Zilza     80 Laura ~ Já h~ Rua Jos~ Barão~
##  6 08/0~ Sim        Não      Lourd~    80 Freder~ Novo  Rua Eur~ Barão~
##  7 09/0~ Sim        Não      Vilan~   100 Eloise~ Já h~ Prof Du~ Barão~
##  8 10/0~ Sim        Não      Vilan~   130 Karim ~ Já h~ Arthur ~ Barão~
##  9 10/0~ Sim        Não      Lourd~   150 Thiago~ Já h~ Rua  Ro~ Barão~
## 10 11/0~ Sim        Não      Zilza    170 Anders~ Já h~ Rua  De~ Barão~
## # ... with 6 more variables: `Feedback Colhido?` <chr>, `Onde foi
## #   colhido?` <chr>, `Nota feedback mulher` <dbl>, `Nota feedback
## #   cliente` <dbl>, `Feedback cliente` <chr>, Comentários <chr>
\end{verbatim}

\hypertarget{planilha-de-faxinas-do-aplicativo}{%
\subsubsection{Planilha de faxinas do
aplicativo:}\label{planilha-de-faxinas-do-aplicativo}}

\begin{verbatim}
## # A tibble: 10 x 15
##    Data  Colaboradora Valor Cliente Endereço `Ocorreu?` `Feedback Colhi~
##    <chr> <chr>        <dbl> <chr>   <chr>    <chr>      <chr>           
##  1 03/0~ Lourdes         80 Eloise~ Prof Du~ Sim        Não             
##  2 12/0~ Zilza          130 Prof S~ <NA>     Sim        Sim             
##  3 22/0~ Lourdes        170 Derick  <NA>     Sim        Sim             
##  4 08/0~ Zilza          150 Laís Z~ <NA>     Sim        Sim             
##  5 12/0~ Lourdes         80 <NA>    <NA>     Sim        Sim             
##  6 16/0~ Zilza          130 <NA>    <NA>     Sim        Sim             
##  7 17/0~ Zilza          130 <NA>    <NA>     Sim        Sim             
##  8 18/0~ Lourdes         80 <NA>    <NA>     Sim        Sim             
##  9 23/0~ Zilza          150 <NA>    <NA>     Sim        Sim             
## 10 26/0~ Zilza           80 Renata~ Rua  Dr~ Sim        Sim             
## # ... with 8 more variables: `Onde foi colhido?` <chr>, `Nota feedback
## #   cliente` <dbl>, `Nota feedback mulher` <dbl>, Comentários <chr>,
## #   Remarcou <chr>, Tipo <chr>, Região <chr>, `Feedback cliente` <chr>
\end{verbatim}

\hypertarget{planilha-de-disponibilidade}{%
\subsubsection{Planilha de
disponibilidade}\label{planilha-de-disponibilidade}}

\begin{verbatim}
## # A tibble: 10 x 3
##    Colaboradora Disponibilidade Data      
##    <chr>                  <dbl> <chr>     
##  1 Lourdes                    2 01/04/2018
##  2 Lourdes                    2 02/04/2018
##  3 Lourdes                    2 03/04/2018
##  4 Lourdes                    2 04/04/2018
##  5 Lourdes                    2 05/04/2018
##  6 Lourdes                    2 06/04/2018
##  7 Lourdes                    2 07/04/2018
##  8 Lourdes                    2 08/04/2018
##  9 Lourdes                    2 09/04/2018
## 10 Lourdes                    2 10/04/2018
\end{verbatim}

\hypertarget{informauxe7uxf5es-faxineiras}{%
\subsubsection{Informações
faxineiras}\label{informauxe7uxf5es-faxineiras}}

\hypertarget{informauxe7uxf5es-clientes}{%
\subsubsection{Informações Clientes}\label{informauxe7uxf5es-clientes}}

\end{document}
